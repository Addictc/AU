\documentclass[12pt,a4paper]{article}
\title{Calculus $\beta$ 2b}
\author{Rasmus Crolly}
\date{\today}

\usepackage[utf8]{inputenc}
\usepackage{graphicx} %for images%
\usepackage{pgfplots} %for plotting in latex%
\usepackage{amsmath}  %math font%
\usepackage{amssymb}
\usepackage{siunitx} %si unit stuff%
\usepackage{biblatex}
\addbibresource{sources.bib}

\pgfplotsset{compat=1.16}
\usetikzlibrary{external}
\tikzexternalize[prefix=tikz/]

\newcommand{\R}{\mathbb{R}}

\setlength{\parindent}{0pt}

\begin{document}
\maketitle

\section{2.12)}


\subsection{introduction}
For a function 
\begin{math}
f : D(f)\to\R
\end{math}
the range is the values that \begin{math}
f
\end{math}
hits.
A real number \begin{math}
t
\end{math}
is said to be in the range of 
\begin{math}
f
\end{math}
if there exists an element \begin{math}
z \in D(f)
\end{math}
where
\begin{math}
t=f(z)
\end{math}
(from 2.1 "Terminologi og notation" in the book) \cite{calcbeta2.1}


\subsection{a}
\begin{equation} 
f : \R ^2\to \R 
\end{equation}
where
\begin{equation} 
f(x,y)=x-y
\end{equation}

Since both x and y can assume any real value the range of the function is

\begin{equation} 
Vm(f)=\left(-\infty,\infty\right)={t\in\R : -\infty < t < \infty }
\end{equation}

Note that the use of soft brackets is because infinity is not technically a real number and therefore not within the range

\subsection{b}
\begin{equation} 
f : \R ^2 \to \R 
\end{equation}
where
\begin{equation} 
f(x,y)=x^2+y^2
\end{equation}
Since both of the variables have 2 as exponent neither of them can ever return a value below 0 and since they are added together they can return any positive value or 0

\begin{equation} 
Vm(f)=\left[0,\infty\right)=\{t\in\R : 0 < t < \infty \} 
\end{equation}

\subsection{c}


\begin{math} 
f : D(f)\to\R  
\end{math}
given that
\begin{math} 
f(x,y)=\frac{1}{x^2+y^2}
\end{math}
,
where
\begin{equation} 
D(f)={(x,y) \in \R^2 : (x,y) \neq (0,0)}
\end{equation}

When
\begin{equation}
\lim_{x \land y \to 0} f(x,y) \Rightarrow \lim_{f\to\infty} 
\end{equation}
When 
 \begin{equation}
\lim_{x \lor y \to \infty} f(x,y) \lim_{f\to0} 
\end{equation}


\begin{equation} 
Vm(f)=\left(0,\infty\right) =\{t\in\R : 0 < t < \infty \} 
\end{equation}




\subsection{d}
\begin{center}
\begin{math}
f : \R ^2\to \R 
\end{math}
where
\begin{math} 
f(x,y)=\frac{7}{x^2+y^2+1}
\end{math}
\end{center}
\vspace{2 mm} 

We first observe that the function is continious as per Sentence (1.16) \cite{calcbeta1.16} in the book 7 and 1 are both continious functions, as is 
\begin{math}
x^2
\end{math}
and
\begin{math}
y^2
\end{math}
and since both 
\begin{math}
x^2
\end{math}
and
\begin{math}
y^2
\end{math}
will return positive numbers or 0 the denominator must be different from 0 since it also contains a +1

\vspace{2 mm} 
If both x and y are 0 the function will return 7, when they are very large the function will tend towards 0 and the range of the function is therefore

\begin{equation} 
Vm(f)=\left(0,7\right] 
\end{equation}



\printbibliography
\end{document}
