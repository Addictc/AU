
\documentclass[12pt,a4paper]{article}
\title{Mechanics and Thermodynamics}
\author{Anon}
\date{\today}

\usepackage[utf8]{inputenc}
\usepackage{graphicx}
\usepackage{pgfplots}
\usepackage{amsmath}
\usepackage{amssymb}
\usepackage[T1]{fontenc}
\usepackage{textcomp}

\usepackage[
backend=biber,
style=alphabetic,
sorting=ynt
]{biblatex}
\addbibresource{sources.bib}

\usepackage{siunitx}
\usepackage[cm]{fullpage}
\usepackage{listings}

\pgfplotsset{compat=1.16}
\usetikzlibrary{external}
\tikzexternalize[prefix=tikz/]

\newcommand{\R}{\mathbb{R}}

\setlength{\parindent}{0pt}

\usepackage{hyperref}
\hypersetup{
colorlinks=true,
linkcolor=blue,
filecolor=magenta,
urlcolor=cyan,
}				
\urlstyle{same}

% figure support
\usepackage{import}
\usepackage{xifthen}
\pdfminorversion=7
\usepackage{pdfpages}
\usepackage{transparent}
\newcommand{\incfig}[1]{%
	\def\svgwidth{\columnwidth}
	\import{./figures/}{#1.pdf_tex}
}

\pdfsuppresswarningpagegroup=1

\begin{document}
\maketitle
\section{Objects on a ring}

\subsection{1}

\subsubsection{i}

In the initial stage the heavier object has a kinetic energy of 0 since it is at rest.

The potential energy in the initial position consists of two parts

\begin{itemize}
	\item potential energy due to gravity
	\item potential energy due to the pull of the spring
\end{itemize}


Potential energy due to gravity is calculated by \[
	E_{grav}=mgh
.\] 
where $m$ is the mass of the object,  $g$ is the gravitational pull and $h$ is the height

Gravitational potential energy can therefore be expressed as \[
3m\cdot g\cdot 2R = 6mgR
.\] 

The potential energy stored in a spring is expressed as \[
U_{el} = \frac{1}{2}kx^2
.\] 

Where the $x$-axis has the same direction as the spring. The lenght of  $x$ can be found by using pythagoras theorem \[
a^{2}+b^{2}=c^{2}
.\] 
$c$ and therefore $x$ is thus \[
\sqrt{R^{2}+2R^{2}} 
.\] 

resulting in the final expression for $U_{el}$ \[
	U_{el}=\frac{1}{2}k\left( R^2+2R^2 \right)
.\] 

By adding the expressions for $U_{el}$ and $E_{grav}$ the expression for the total potential energy in state i can be found \[
	E_{i}=6mgR+\frac{1}{2}k\left( R^{2}+2R^{2} \right) 
.\] 

\subsubsection{ii}

in the second state the energy stored in the spring is given by

\[
U_{el}&=\frac{1}{2}kR^2
.\] 
the work done is the difference in stored energy


\[
	W_{el}=\frac{1}{2}k\left( (2R)^2+R^2 \right) -\frac{1}{2}kR^2=2kR^2
.\] 
\[
	W_{grav}=-3mg(2R-R)
.\] 


total work done 

\[
	W_{tot,ii}=2kR^2+3mgR
.\] 

the kinetic energy is therefore 

\[
2kR^2+3mgR= E_{kin}
.\] 



\subsection{iii} 

Since the spring length is the same as in the initial state the difference in potential energy is the difference in kinetic energy 

E_{pot,iii}=0

E_{kin,iii}=E_{pot,i}


\subsection{b} 

the velocity is found by

\[
E_{kin}=\frac{1}{2}mv^2
.\] 

Since the kinetic energy in iii is the same as the potential energy in state i 
\[
	6mgR&=\frac{1}{2}3mv^2 
.\] 
and solving for v

\begin{align*}
	12mgR=3mv^2\\
4mgR=mv^2\\
4gR=v^2\\
\sqrt{4gR}=v
.\end{align*}

their joint velocity can be found since there is conservation of impluse

\[
m_1*v_1=m_2*v_2
.\] 

\[
12mgR=4m*v_2
.\] 
solving for $v_2$

\[
v_2=3gR
.\] 



\subsection{c} 


to reach the point A, the object has to overcome the force of gravity by adding the energy required to reach A

\[
E_{grav}=4mgR
.\] 





































\end{document}
