
\documentclass[12pt,a4paper]{article}
\title{Calculus $\beta$}
\author{Rasmus Crolly}
\date{\today}


\usepackage{graphicx}
\usepackage{pgfplots}
\usepackage{amsmath}
\usepackage{amssymb}
\usepackage[T1]{fontenc}
\usepackage{textcomp}

\usepackage[
backend=biber,
style=alphabetic,
sorting=ynt
]{biblatex}
\addbibresource{sources.bib}

\usepackage{siunitx}
\usepackage[cm]{fullpage}
\usepackage{listings}

\pgfplotsset{compat=1.16}
\usetikzlibrary{external}
\tikzexternalize[prefix=tikz/]

\newcommand{\R}{\mathbb{R}}

\setlength{\parindent}{0pt}

\usepackage{hyperref}
\hypersetup{
colorlinks=true,
linkcolor=blue,
filecolor=magenta,
urlcolor=cyan,
}				
\urlstyle{same}

% figure support
\usepackage{import}
\usepackage{xifthen}
\pdfminorversion=7
\usepackage{pdfpages}
\usepackage{transparent}
\newcommand{\incfig}[1]{%
	\def\svgwidth{\columnwidth}
	\import{./figures/}{#1.pdf_tex}
}

\pdfsuppresswarningpagegroup=1

\begin{document}
\maketitle
	
\section{u12}

given the function \[
	f\left( x,y \right) = 4x^2y+5xy^2+x^3
.\] 

\subsection{a}


$f_{y}$ is determined by differentiating with respect to y \[
f_{y}=4x^2+10xy
.\] 

\subsection{b}
the gradient is found by finding the partially derived with respect to x and y respectively. since $f_{y}$ is already found in a) the only one left to find is \[
f_{x}=8xy+5y^2+3x^2
.\] 
the gradient is therefore \[
\nabla f(x,y) = (8xy+5y^2+3x^2,4x^2+10xy) 
.\] 



\subsection{c}
calculating the unit vector from the given vector by using pythagoras






\subsection{d}



\subsection{e}




\end{document}
