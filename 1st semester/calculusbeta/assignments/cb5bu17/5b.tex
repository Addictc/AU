
\documentclass[12pt,a4paper]{article}
\title{Calculus $\beta$ or Mechanics and Thermodynamics}
\author{Anon}
\date{\today}


\usepackage{graphicx}
\usepackage{pgfplots}
\usepackage{amsmath}
\usepackage{amssymb}
\usepackage[T1]{fontenc}
\usepackage{textcomp}

\usepackage[
backend=biber,
style=alphabetic,
sorting=ynt
]{biblatex}
\addbibresource{sources.bib}

\usepackage{siunitx}
\usepackage[cm]{fullpage}
\usepackage{listings}

\pgfplotsset{compat=1.16}
\usetikzlibrary{external}
\tikzexternalize[prefix=tikz/]

\newcommand{\R}{\mathbb{R}}

\setlength{\parindent}{0pt}

\usepackage{hyperref}
\hypersetup{
colorlinks=true,
linkcolor=blue,
filecolor=magenta,
urlcolor=cyan,
}				
\urlstyle{same}

% figure support
\usepackage{import}
\usepackage{xifthen}
\pdfminorversion=7
\usepackage{pdfpages}
\usepackage{transparent}
\newcommand{\incfig}[1]{%
	\def\svgwidth{\columnwidth}
	\import{./figures/}{#1.pdf_tex}
}

\pdfsuppresswarningpagegroup=1

\begin{document}
\maketitle
	



\section{U17} 

The function in question is \[
	f(x,y)=2x^3+y^2-24x-6y+5
.\] 


\section{a} 

Finding the partial derivatives by differentiating with respect to x and y respectively

\[
f_{x} = 6x^2-24
.\] 
and

\[
f_{y} = 2y-6
.\] 


\section{b}

since the objective is to find critical points the gradient must be $\nabla f(x,y) = (0, 0)$ which results in the following two equations
\begin{align*}
	6x^2-24&=0\\ 2y-6&=0
.\end{align*}

Since they only happen to contain one variable each they can easily be solved for their respective variable

\begin{align*}
	x=\sqrt{4}\\
	y=3
.\end{align*}

$y_0$ is therefore 3


\section{c}

The critical value is found by inserting the critical points in the function

\[
	f(2,3)=2*2^3+3^2-24*2-6*3+5= -36
.\] 
and
\[
	f(-2,3)=2(-2)^3+3^2-24*(-2)-6*3+5=28
.\] 
the greatest critical value is therefore 28


\section{d} 

\[
	(x_1,y_0)=(-2,3)
.\] 

the partial derivatives of second order of f is


\[
\frac{\partial^2 }{\partial^2 x}=12x\\\frac{\partial^2 }{\partial^2 y} = 2 
.\] 

and inserting $(x_1,y_0)$ 

\[
\frac{\partial^2 }{\partial^2 x} &=-24\\\frac{\partial^2 }{\partial^2 y} &=2
.\] 


\section{e} 

The determinant of a matrix 
\[
\begin{bmatrix}
	 a & b \\ c & d 
\end{bmatrix}
.\] 
is defined as
\[
D=ad-bc
.\] 

For the Hessian matrix used in the second order criteria
\[
	H_{f}=\begin{bmatrix}  \frac{\partial^2f}{\partial x^2}(x_0,y_0) & \frac{\partial^2f}{\partial x,\partial y}(x_0,y_0)\\\frac{\partial^2 f}{\partial x, \partial y}(x_0,y_0) & \frac{\partial^2 f}{\partial y^2}(x_0,y_0) \end{bmatrix} 
.\] 

\[
	D=\frac{\partial^2f}{\partial x^2}(x_0,y_0)\frac{\partial^2 f}{\partial y^2}(x_0,y_0)-\left( \frac{\partial^2f}{\partial x,\partial y}(x_0,y_0) \right) ^2
.\] 

 $\frac{\partial^2f}{\partial x,\partial y} $ of the function in this case is \[
 \frac{\partial^2f}{\partial x,\partial y} = 0
 .\] 

 The determinant of the Hessian matrix in the point $(-2,3)$ is therefore
  \[
 D=-48
 .\] 

\section{f}

According to the result in section e the critical point $(-2,3)$ is a saddle point since  $d<0$

\section{g} 

Since inserting $y=\frac{5}{2}$ in $f_{y}$ returns -1, x must be 0 otherwise Pythagoras would result in a length greater than 1 which would mean it is no longer a unit vector
\[
	1^2=\sqrt{x^2+(-1)^2} 
.\] 



\end{document}
