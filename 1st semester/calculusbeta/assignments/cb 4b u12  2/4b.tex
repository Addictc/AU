
\documentclass[12pt,a4paper]{article}
\title{Calculus $\beta$}
\author{Rasmus Crolly}
\date{\today}


\usepackage{graphicx}
\usepackage{pgfplots}
\usepackage{amsmath}
\usepackage{amssymb}
\usepackage[T1]{fontenc}
\usepackage{textcomp}

\usepackage[
backend=biber,
style=alphabetic,
sorting=ynt
]{biblatex}
\addbibresource{sources.bib}

\usepackage{siunitx}
\usepackage[cm]{fullpage}
\usepackage{listings}

\pgfplotsset{compat=1.16}
\usetikzlibrary{external}
\tikzexternalize[prefix=tikz/]

\newcommand{\R}{\mathbb{R}}

\setlength{\parindent}{0pt}

\usepackage{hyperref}
\hypersetup{
colorlinks=true,
linkcolor=blue,
filecolor=magenta,
urlcolor=cyan,
}				
\urlstyle{same}

% figure support
\usepackage{import}
\usepackage{xifthen}
\pdfminorversion=7
\usepackage{pdfpages}
\usepackage{transparent}
\newcommand{\incfig}[1]{%
	\def\svgwidth{\columnwidth}
	\import{./figures/}{#1.PDF_Tex}
}

\pdfsuppresswarningpagegroup=1

\begin{document}
\maketitle
	
\section{up}

given the function \[
	f\left( x,y \right) = 4x^2y+5xy^2+x^3
.\] 

\subsection{a}


$f_{y}$ is determined by differentiating with respect to y \[
f_{y}=4x^2+10xy
.\] 

\subsection{b}
the gradient is found by finding the partially derived with respect to x and y respectively. Since $f_{y}$ is already found in a) the only one left to find is \[
f_{x}=8xy+5y^2+3x^2
.\] 
the gradient is therefore \[
\nabla f(x,y) = (8xy+5y^2+3x^2,4x^2+10xy) 
.\] 



\subsection{c}
calculating the unit vector from the given vector by using Pythagoras to find the length of the vector

\[
	 \mid \vec{v} \mid   =\sqrt{8^2+(-6)^2} = 10
.\] 

then dividing the vector $\vec{v}$ by its length to get the unit vector in the direction \[
\overline{u}=\begin{pmatrix} 0.8\\ -0.6 \end{pmatrix}
.\] 


\subsection{d}

according to sentence 3.10:

\[
	D_{\overline{u}}f(x,y)=\frac{\partial f}{\partial x} (x,y)a+\frac{\partial f}{\partial y} (x,y)b
.\]
where a and b are the x and y values of the unit vector

since the partial derivatives are already known from the gradient the values can all be inserted which results in

\[
	D_{\overline{u}}f(2,1)=(3\cdot 2^2 + 8\cdot 2\cdot 1 + 5\cdot 1^2)\cdot 0.8+( 4\cdot 2^2 + 10\cdot 2\cdot 1)\cdot  (-0.6)=0.48
.\] 



\subsection{e}

the greatest directional derivative in a point is the same as the length of the gradient in the point which means using Pythagoras 

\[
 \mid \overline{\nabla } \mid = \sqrt{\left(4\cdot 2^{2} + 10 \cdot 2\cdot  1\right)^{2} + \left(3 \cdot 2^{2} + 8\cdot 2\cdot 1 + 5\cdot  1^{2}\right)^{2}}=\sqrt{36^2+33^2}=\sqrt{2385}  
.\] 

by dividing the gradient by the length the unit vector $\overline{v}$ is found \[
\begin{pmatrix} \frac{33}{ \sqrt{2385}}\\ \frac{36}{ \sqrt{2385}} \end{pmatrix}
.\] 

\[
\overline{v}= \begin{pmatrix} 11\frac{\sqrt{265}}{265}\\
12\frac{\sqrt{265}}{265} \end{pmatrix}
.\] 

\subsection{f} 

the value of the greatest directional derivative is equivalent to the length of the gradient
\[
  \mid \overline{\nabla } \mid = \sqrt{\left(4\cdot 2^{2} + 10 \cdot 2\cdot  1\right)^{2} + \left    (3 \cdot 2^{2} + 8\cdot 2\cdot 1 + 5\cdot  1^{2}\right)^{2}}=\sqrt{36^2+33^2}=\sqrt{2385} 
.\] 



\end{document}
