
\documentclass[12pt,a4paper]{article}
\title{Mechanics and Thermodynamics}
\author{Anon}
\date{\today}

\usepackage[utf8]{inputenc}
\usepackage{graphicx}
\usepackage{pgfplots}
\usepackage{amsmath}
\usepackage{amssymb}
\usepackage[T1]{fontenc}
\usepackage{textcomp}

\usepackage[
backend=biber,
style=alphabetic,
sorting=ynt
]{biblatex}
\addbibresource{sources.bib}

\usepackage{siunitx}
\usepackage[cm]{fullpage}
\usepackage{listings}

\pgfplotsset{compat=1.16}
\usetikzlibrary{external}
\tikzexternalize[prefix=tikz/]

\newcommand{\R}{\mathbb{R}}

\setlength{\parindent}{0pt}

\usepackage{hyperref}
\hypersetup{
colorlinks=true,
linkcolor=blue,
filecolor=magenta,
urlcolor=cyan,
}				
\urlstyle{same}

% figure support
\usepackage{import}
\usepackage{xifthen}
\pdfminorversion=7
\usepackage{pdfpages}
\usepackage{transparent}
\newcommand{\incfig}[1]{%
	\def\svgwidth{\columnwidth}
	\import{./figures/}{#1.pdf_tex}
}

\pdfsuppresswarningpagegroup=1

\begin{document}
\maketitle
\section{Objects on a ring}

\subsection{1}

\subsubsection{i}

In the initial stage the heavier object has a kinetic energy of 0 since it is at rest.

The potential energy in the initial position consists of two parts

\begin{itemize}
	\item potential energy due to gravity
	\item potential energy due to the pull of the spring

\end{itemize}


Potential energy due to gravity is calculated by \[
	E_{grav}=mgh
.\] 
where $m$ is the mass of the object,  $g$ is the gravitational pull (assumed to be $9.8\unit{m.s^{-2}}$) and $h$ is the














\end{document}
